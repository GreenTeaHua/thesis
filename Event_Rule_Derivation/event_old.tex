\documentclass{article}
	
\usepackage{thesis_style}

\bibliographystyle{ieeetr}
\graphicspath{{./img/}}
\newcommand\scalemath[2]{\scalebox{#1}{\mbox{\ensuremath{\displaystyle #2}}}}
\begin{document}
	\title{Derivation of an event-triggering rule for the fast and saturation controller}
	\author{Diogo Almeida}
	\maketitle	
	
	\section{The problem}
		We have a continuous controller proposed in \cite{lohmann_attitude}, where a Lyapunov function $V(\mathbf{x})$ is defined, as well as a resulting control signal, $\boldsymbol \tau(\mathbf{x})$, defined as
		\begin{equation}
			\boldsymbol \tau(\mathbf{x}) = \mathbf{T(q)} - \mathbf{D(x)} \boldsymbol \omega.
			\label{tau_cont}
		\end{equation}
		
		Where the state $\mathbf{x}$ is the system attitude, $\mathbf{x} = \left [\mathbf{q} \; \boldsymbol \omega \right]$, and $V(\mathbf{x})$ is given by the sum of the kinetic energy of the system, $E_{rot}(\boldsymbol \omega)$, with an artificial potential energy $E_{pot}(\mathbf{q})$. Its time derivative is
		\[
			\dot V(\mathbf{x}) = \boldsymbol \omega^\top \boldsymbol \tau(\mathbf{x}) - \mathbf{T(q)}^\top \boldsymbol \omega
		\]
		and, by applying \eqref{tau_cont}, we get
		\begin{equation}
			\dot V(\mathbf{x}) = -\boldsymbol \omega^\top \mathbf{D(x)} \boldsymbol \omega,
			\label{vdot_cont}
		\end{equation}
		where $\mathbf{D(x)}$ is a positive semi-definite matrix, ensuring $\dot V(\mathbf{x}) \leq 0$.
		
		The challenge now is to find an event-triggering rule that allows for a non-periodic update of the control signal and at the same time does not permit the derivative of the Lyapunov function to grow above zero.
	\section{Definitions}
		\subsection{State error}
			Assuming that the last sampling instant is given by $t_k$, the state evolution can be given by $\mathbf{x}(t) = \mathbf{x}(t_k) + \mathbf{e}(t)$, where the error $\mathbf{e}$ is defined as
			\begin{equation}
				\mathbf{e}(t) = \mathbf{x}(t) - \mathbf{x}(t_k) = \begin{bmatrix}
											\mathbf{q}(t) - \mathbf{q}(t_k)\\
											\boldsymbol \omega(t) - \boldsymbol \omega(t_k)
										  \end{bmatrix} =			
										 \begin{bmatrix}
											 \mathbf{\hat{q}} \\
											  \hat{\boldsymbol \omega}
										  \end{bmatrix}
				\label{error}
			\end{equation}
		
		\subsection{Attitude quaternion}
			The proposed controler parameterizes the attitude using a quaternion, $\mathbf{q}$, that simbolizes the attitude error with respect to a given reference. Since it prioritizes the thrust direction alignment over the yaw, it further decomposes the quaternion into the product of two other quaternions, $\mathbf{q_{xy}} = \left [q_x \; q_y \; 0 \; q_p \right]^\top$  and $\mathbf{q_{z}} = \left[0\;0\;q_z\;q_w \right]$. From these two quaternions we can extract the displacement angle of the thrust axis, $\varphi = 2 \arccos(q_p)$ and the yaw error angle, $\vartheta = 2 \arccos(q_w)$. 
		
		\subsection{Auxiliary function}
			Several auxiliary functions are proposed in \cite{lohmann_attitude}. In particular, $\Lambda_{\epsilon_l}^{\epsilon_u}(\epsilon)$ is of interest. It is defined as
			\begin{equation}
				\Lambda_{\epsilon_l}^{\epsilon_u}(\epsilon)
				\label{lambda}
			\end{equation}
		\subsection{Artificial torque field}
			The torque field generated by the artificial potencial energy, $\mathbf{T(q)}$, is given by the sum of four fields, 
			\begin{equation}
				\mathbf{T(q)} = \mathbf{T_\varphi^\varphi(q)} + \mathbf{T_\varphi^\vartheta(q)} + \mathbf{T_\perp^\vartheta(q)} + \mathbf{T_z^\vartheta(q)}
				\label{torques}
			\end{equation}
			where
			\begin{eqnarray*}
				\mathbf{T_\varphi^\varphi(q)}  =& \displaystyle \frac{\displaystyle c_\varphi \Lambda_{\varphi_l}^{\varphi_u}(\varphi)}{\displaystyle\sqrt{1-q_p^2}} &\begin{bmatrix}
																		q_x\\
																		q_y\\
																		0
																	    \end{bmatrix}\\
				\mathbf{T_\varphi^\vartheta(q)}  =& -\displaystyle \frac{\displaystyle c_\vartheta \cos^3\left (\frac{\varphi}{2} \right)\sin \left(\frac{\varphi}{2} \right) \displaystyle \int_0^\vartheta{\Lambda_{\vartheta_l}^{\vartheta_u}(\epsilon)} d\epsilon }{\displaystyle\sqrt{1-q_p^2}} &\begin{bmatrix}
										q_x\\
										q_y\\
										0
									    \end{bmatrix}\\
				\mathbf{T_\perp^\vartheta(q)}  =& -\displaystyle \frac{\displaystyle q_z c_\vartheta \cos^3 \left(\frac{\varphi}{2} \right)\sin \left(\frac{\varphi}{2}\right) \Lambda_{\vartheta_l}^{\vartheta_u}(\vartheta)}{\displaystyle\sqrt{\displaystyle 1-q_w^2} \sqrt{\displaystyle 1-q_p^2}} &\begin{bmatrix}
										q_y\\
										-q_x\\
										0
									    \end{bmatrix}\\
				\mathbf{T_z^\vartheta(q)}  =& -\displaystyle \frac{\displaystyle q_z c_\vartheta \cos^4 \left(\frac{\varphi}{2} \right) \Lambda_{\vartheta_l}^{\vartheta_u}(\vartheta)}{\displaystyle \sqrt{\displaystyle 1-q_w^2} } &\begin{bmatrix}
											 0\\
											 0\\
											 1
										       \end{bmatrix}
			\end{eqnarray*}
		 	and, since 
		 	\begin{eqnarray*}
			 	\sin \left(\displaystyle \frac{\varphi}{2} \right) &=& \sqrt{1-q_p^2}\\
			 	\cos^3 \left(\displaystyle\frac{\varphi}{2} \right) &=& q_p^3\\
			 	\cos^4 \left(\displaystyle\frac{\varphi}{2} \right) &=& q_p^4
		 	\end{eqnarray*}
		 	the torque field \eqref{torques} becomes
		 	\begin{equation}
		 		\mathbf{T(q)} = \begin{bmatrix}
		 					\displaystyle \left ( c_\varphi \Lambda_{\varphi_l}^{\varphi_u}(\varphi) - q_p^3 c_\vartheta \int_0^\vartheta{\Lambda_{\vartheta_l}^{\vartheta_u}(\epsilon)} d\epsilon \right ) q_x + \frac{q_z q_p^3 c_\vartheta \Lambda_{\vartheta_l}^{\vartheta_u}(\vartheta) q_y}{\sqrt{1-q_w^2}}\\ \\
		 					\displaystyle \left ( c_\varphi \Lambda_{\varphi_l}^{\varphi_u}(\varphi) - q_p^3 c_\vartheta \int_0^\vartheta{\Lambda_{\vartheta_l}^{\vartheta_u}(\epsilon)} d\epsilon \right ) q_y - \frac{q_z q_p^3 c_\vartheta \Lambda_{\vartheta_l}^{\vartheta_u}(\vartheta) q_x}{\sqrt{1-q_w^2}}\\ \\
							\displaystyle \frac{q_z q_p^4 c_\vartheta \Lambda_{\vartheta_l}^{\vartheta_u}(\vartheta)}{\sqrt{1-q_w^2}}	 		
		 				\end{bmatrix}
		 		\label{torques_vec}
		 	\end{equation}

	\section{Rule derivation}
		Using an event-triggering rule, $\boldsymbol \tau(t) = \boldsymbol \tau(t_k)$, equation \eqref{vdot_cont} does not hold anymore. Instead, we have
		\[
			\dot V(\mathbf{x}(t)) = \boldsymbol \omega (t)^\top \boldsymbol \tau(t_k) - \mathbf{T(q}(t))^\top\boldsymbol \omega(t)
		\]
		and, by recalling $\mathbf{x}(t) = \mathbf{x}(t_k) + \mathbf{e}(t)$, together with \eqref{error} and \eqref{tau_cont}
		\[
			\dot V(\mathbf{x}(t)) = \boldsymbol \omega(t)^\top \left [ \mathbf{T(q}(t_k)) - \mathbf{T(q}(t_k) + \mathbf{\hat{q}}(t)) \right ] - \boldsymbol \omega(t_k)^\top \mathbf{D(x}(t_k)) \boldsymbol \omega(t_k) -  \hat{\boldsymbol \omega}(t)^\top \mathbf{D(x}(t_k)) \boldsymbol \omega(t_k).
		\]
		
		The term $-\boldsymbol \omega(t_k)^\top \mathbf{D(x}(t_k)) \boldsymbol \omega(t_k) = \dot V(\mathbf{x}(t_k))$ is always $\leq 0$, as it is constant and obtained as in \cite{lohmann_attitude}. $\hat{\boldsymbol \omega}(t)^\top \mathbf{D(x}(t_k)) \boldsymbol \omega(t_k)$ has $\hat{\boldsymbol \omega}(t)$ as the only variable, making it easy to monitor. The subtraction $\mathbf{T(q}(t_k)) - \mathbf{T(q}(t_k) + \mathbf{\hat{q}}(t))$ is the operation that is harder to compute, since it involves constantly doing the computations \ref{torques_vec}. It is possible to linearize $\mathbf{T(q)}$ around $\mathbf{q}(t_k)$ to obtain an expression that is linear in an error term:
		\begin{equation}
			\mathbf{T(q}(t)) \simeq \mathbf{T(q}(t_k)) +  \nabla \mathbf{T(q}(t)) 
			\label{lin_torque} 
		\end{equation}
		where
		\[
			\displaystyle \nabla \mathbf{T(q}(t)) =\left .\frac{\partial T}{\partial q_x}\right |_{t=t_k}\hat{q_x}(t) +
								\left .\frac{\partial T}{\partial q_y}\right |_{t=t_k}\hat{q_y}(t) +
								\left .\frac{\partial T}{\partial q_z}\right |_{t=t_k}\hat{q_z}(t) +
								\left .\frac{\partial T}{\partial q_p}\right |_{t=t_k}\hat{q_p}(t) +
								\left .\frac{\partial T}{\partial q_w}\right |_{t=t_k}\hat{q_w}(t)				
		\]
		Using \eqref{lin_torque}, $\dot V(\mathbf{x}(t))$ becomes
		\[
			\dot V(\mathbf{x}(t)) = -\boldsymbol \omega(t)^\top  \nabla \mathbf{T(q}(t_k)) + \dot V(\mathbf{x}(t_k)) - \hat{\boldsymbol \omega}(t)^\top \mathbf{D(x}(t_k)) \boldsymbol \omega(t_k).
		\]
		
		By ensuring
		\begin{equation}
			\boldsymbol \omega(t)^\top  \nabla \mathbf{T(q}(t)) + \hat{\boldsymbol \omega}(t)^\top \mathbf{D(x}(t_k)) \boldsymbol \omega(t_k) \leq \dot V(\mathbf{x}(t_k)) 
			\label{trigger}
		\end{equation}
		we get $\dot V(\mathbf{x}(t)) \leq 0$. We need to compute the partial derivatives that composes $\nabla \mathbf{T(q}(t))$ (constant in between sampling instants) in order to apply \eqref{trigger}:
		\begin{eqnarray*}
				\left .\frac{\partial T}{\partial q_x}\right |_{t=t_k}  =& \begin{bmatrix}
												c_\varphi \Lambda_{\varphi_l}^{\varphi_u}(\varphi(t_k)) - q_p^3(t_k) c_\vartheta \displaystyle \int_0^{\vartheta(t_k)}{\Lambda_{\vartheta_l}^{\vartheta_u}(\epsilon)} d\epsilon \\\\
												-\displaystyle \frac{q_z(t_k) q_p^3(t_k) c_\vartheta \Lambda_{\vartheta_l}^{\vartheta_u}(\vartheta(t_k))}{\sqrt{1-q_w^2(t_k)}}\\\\
												0
											   \end{bmatrix}
											   \\
				\left .\frac{\partial T}{\partial q_y}\right |_{t=t_k}  =& \begin{bmatrix}
												\displaystyle\frac{c_\vartheta q_z(t_k) q_p^3(t_k) \Lambda_{\vartheta_l}^{\vartheta_u}(\vartheta(t_k))}{\sqrt{1-q_w^2(t_k)}} \\\\
												c_\varphi \Lambda_{\varphi_l}^{\varphi_u}(\varphi(t_k)) - q_p^3(t_k) c_\vartheta \displaystyle \int_0^{\vartheta(t_k)}{\Lambda_{\vartheta_l}^{\vartheta_u}(\epsilon)} d\epsilon\\\\
												0
											   \end{bmatrix}
											   \\
				\left .\frac{\partial T}{\partial q_z}\right |_{t=t_k}  =& \begin{bmatrix}
												\displaystyle \frac{c_\vartheta q_y(t_k) q_p^3(t_k) \Lambda_{\vartheta_l}^{\vartheta_u}(\vartheta(t_k))}{\sqrt{1-q_w^2(t_k)}} \\\\
												\displaystyle \frac{c_\vartheta q_x(t_k) q_p^3(t_k) \Lambda_{\vartheta_l}^{\vartheta_u}(\vartheta(t_k))}{\sqrt{1-q_w^2(t_k)}} \\\\
												\displaystyle \frac{c_\vartheta q_p^4(t_k) \Lambda_{\vartheta_l}^{\vartheta_u}(\vartheta(t_k))}{\sqrt{1-q_w^2(t_k)}}
											   \end{bmatrix}
		\end{eqnarray*}
		$\left .\frac{\partial T}{\partial q_p}\right |_{t=t_k}$ and $\left .\frac{\partial T}{\partial q_w}\right |_{t=t_k}$ will depend on the value of $\varphi$ and $\vartheta$, respectively, due to \eqref{lambda}. For $0 \leq \varphi < \varphi_l$, $\frac{\partial \varphi}{\partial q_p} = - \frac{2}{\sqrt{1-q_p^2}}$:
		\[
			\left .\frac{\partial T}{\partial q_p}\right |_{t=t_k}  = \begin{bmatrix}
												\left (-\frac{2 c_\varphi}{\sqrt{1-q_p^2(t_k)}} - 3 q_p^2(t_k) c_\vartheta \displaystyle \int_0^{\vartheta(t_k)}{\Lambda_{\vartheta_l}^{\vartheta_u}(\epsilon)} d\epsilon \right )q_x(t_k) + \displaystyle\frac{3 c_\vartheta q_z(t_k) q_y(t_k) q_p^2(t_k) \Lambda_{\vartheta_l}^{\vartheta_u}(\vartheta(t_k))}{\sqrt{1-q_w^2(t_k)}}  \\\\
												\left (-\frac{2 c_\varphi}{\sqrt{1-q_p^2(t_k)}} - 3 q_p^2(t_k) c_\vartheta \displaystyle \int_0^{\vartheta(t_k)}{\Lambda_{\vartheta_l}^{\vartheta_u}(\epsilon)} d\epsilon \right )q_y(t_k) - \displaystyle\frac{3 c_\vartheta q_z(t_k) q_x(t_k) q_p^2(t_k) \Lambda_{\vartheta_l}^{\vartheta_u}(\vartheta(t_k))}{\sqrt{1-q_w^2(t_k)}} \\\\
												\displaystyle \frac{4 c_\vartheta q_z(t_k) q_p^3(t_k) \Lambda_{\vartheta_l}^{\vartheta_u}(\vartheta(t_k))}{\sqrt{1-q_w^2(t_k)}}
											   \end{bmatrix}
		\] 
		for $\varphi_l \leq \varphi < \varphi_u$:
		\[
			\left .\frac{\partial T}{\partial q_p}\right |_{t=t_k}  = \begin{bmatrix}
												\left (- 3 q_p^2(t_k) c_\vartheta \displaystyle \int_0^{\vartheta(t_k)}{\Lambda_{\vartheta_l}^{\vartheta_u}(\epsilon)} d\epsilon \right )q_x(t_k) + \displaystyle\frac{3 c_\vartheta q_z(t_k) q_y(t_k) q_p^2(t_k) \Lambda_{\vartheta_l}^{\vartheta_u}(\vartheta(t_k))}{\sqrt{1-q_w^2(t_k)}}  \\\\
												\left ( - 3 q_p^2(t_k) c_\vartheta \displaystyle \int_0^{\vartheta(t_k)}{\Lambda_{\vartheta_l}^{\vartheta_u}(\epsilon)} d\epsilon \right )q_y(t_k) - \displaystyle\frac{3 c_\vartheta q_z(t_k) q_x(t_k) q_p^2(t_k) \Lambda_{\vartheta_l}^{\vartheta_u}(\vartheta(t_k))}{\sqrt{1-q_w^2(t_k)}} \\\\
												\displaystyle \frac{4 c_\vartheta q_z(t_k) q_p^3(t_k) \Lambda_{\vartheta_l}^{\vartheta_u}(\vartheta(t_k))}{\sqrt{1-q_w^2(t_k)}}
											   \end{bmatrix}
		\] 
		and, for $\varphi_u \leq \varphi < \pi$:
		\[
			\left .\frac{\partial T}{\partial q_p}\right |_{t=t_k}  = \begin{bmatrix}
												\left (-\frac{2 c_\varphi \varphi_l}{\sqrt{1-q_p^2(t_k)(\varphi_u-\pi)}} - 3 q_p^2(t_k) c_\vartheta \displaystyle \int_0^{\vartheta(t_k)}{\Lambda_{\vartheta_l}^{\vartheta_u}(\epsilon)} d\epsilon \right )q_x(t_k) + \displaystyle\frac{3 c_\vartheta q_z(t_k) q_y(t_k) q_p^2(t_k) \Lambda_{\vartheta_l}^{\vartheta_u}(\vartheta(t_k))}{\sqrt{1-q_w^2(t_k)}}  \\\\
												\left (-\frac{2 c_\varphi \varphi_l}{\sqrt{1-q_p^2(t_k)(\varphi_u-\pi)}} - 3 q_p^2(t_k) c_\vartheta \displaystyle \int_0^{\vartheta(t_k)}{\Lambda_{\vartheta_l}^{\vartheta_u}(\epsilon)} d\epsilon \right )q_y(t_k) - \displaystyle\frac{3 c_\vartheta q_z(t_k) q_x(t_k) q_p^2(t_k) \Lambda_{\vartheta_l}^{\vartheta_u}(\vartheta(t_k))}{\sqrt{1-q_w^2(t_k)}} \\\\
												\displaystyle \frac{4 c_\vartheta q_z(t_k) q_p^3(t_k) \Lambda_{\vartheta_l}^{\vartheta_u}(\vartheta(t_k))}{\sqrt{1-q_w^2(t_k)}}
											   \end{bmatrix}
		\] 
		
		For $\vartheta = 2\arccos(q_w)$ we get $\frac{\partial \vartheta}{\partial q_w} = -\frac{2}{\sqrt{1-q_w^2}}$ and $\frac{\partial (\vartheta)^2}{\partial q_w} = -\frac{8\arccos(q_w)}{\sqrt{1-q_w^2}}$. When $0 \leq \vartheta < \vartheta_l$, $\int_0^{\vartheta(t)}{\Lambda_{\vartheta_l}^{\vartheta_u}(\epsilon)} d\epsilon = \frac{\vartheta^2}{2}$ and, as such:
		\[
			\left .\frac{\partial T}{\partial q_w}\right |_{t=t_k}  = \begin{bmatrix}
											\frac{4 c_{\vartheta}q_p(t_k)^3 \arccos(q_w(t_k)) q_x(t_k)}{\sqrt{1-q_w(t_k)^2}} + \frac{2c_{\vartheta} q_w(t_k) q_z(t_k) q_p(t_k)^3 \arccos(q_w(t_k)) q_y(t_k)}{\sqrt{1-q_w(t_k)^2}^3} - \frac{2 c_\vartheta q_z(t_k) q_p(t_k)^3 q_y(t_k)}{\sqrt{1-q_w(t_k)^2}^2}\\
											\frac{4 c_{\vartheta}q_p(t_k)^3 \arccos(q_w(t_k)) q_y(t_k)}{\sqrt{1-q_w(t_k)^2}} - \frac{2c_{\vartheta} q_w(t_k) q_z(t_k) q_p(t_k)^3 \arccos(q_w(t_k)) q_x(t_k)}{\sqrt{1-q_w(t_k)^2}^3} + \frac{2 c_\vartheta q_z(t_k) q_p(t_k)^3 q_x(t_k)}{\sqrt{1-q_w(t_k)^2}^2}\\
											\frac{2 c_{\vartheta} q_z(t_k) q_p(t_k)^4 q_w(t_k) \arccos(q_w(t_k))}{\sqrt{1-q_w(t_k)^2}^3} - \frac{2c_{\vartheta} q_z(t_k) q_p(t_k)^4}{\sqrt{1-q_w(t_k)^2}^2}
										\end{bmatrix} 
		\]
		
		When $\vartheta_l \leq \vartheta < \vartheta_u$, $\int_0^{\vartheta(t)}{\Lambda_{\vartheta_l}^{\vartheta_u}(\epsilon)} d\epsilon = \frac{\vartheta_l^2}{2} + \vartheta_l \left ( \vartheta-\vartheta_l \right )$:
		
		\[
			\left .\frac{\partial T}{\partial q_w}\right |_{t=t_k}  = \begin{bmatrix}
											\frac{2 c_{\vartheta}q_p(t_k)^3 \vartheta_l q_x(t_k)}{\sqrt{1-q_w(t_k)^2}} + \frac{c_{\vartheta} q_w(t_k) q_z(t_k) q_p(t_k)^3 \vartheta_l \left (2\arccos(q_w(t_k))-\vartheta_l \right ) q_y(t_k)}{\sqrt{1-q_w(t_k)^2}^3} - \frac{2 c_\vartheta q_z(t_k) q_p(t_k)^3 \vartheta_l q_y(t_k)}{\sqrt{1-q_w(t_k)^2}^2}\\
											\frac{2 c_{\vartheta}q_p(t_k)^3 \vartheta_l q_y(t_k)}{\sqrt{1-q_w(t_k)^2}} - \frac{c_{\vartheta} q_w(t_k) q_z(t_k) q_p(t_k)^3 \vartheta_l \left (2\arccos(q_w(t_k))-\vartheta_l \right ) q_x(t_k)}{\sqrt{1-q_w(t_k)^2}^3} + \frac{2 c_\vartheta q_z(t_k) q_p(t_k)^3 \vartheta_l q_x(t_k)}{\sqrt{1-q_w(t_k)^2}^2}\\
											\frac{c_{\vartheta} q_z(t_k) q_p(t_k)^4 q_w(t_k) \vartheta_l \left (2\arccos(q_w(t_k))-\vartheta_l \right )}{\sqrt{1-q_w(t_k)^2}^3} - \frac{2c_{\vartheta} q_z(t_k) q_p(t_k)^4 \vartheta_l}{\sqrt{1-q_w(t_k)^2}^2}
										\end{bmatrix} 
		\]
		
		Finnaly, for $\vartheta_u \leq \vartheta \leq \pi$, the integral becomes $\frac{\vartheta_l^2}{2} + \vartheta_l \left(\vartheta_u-\vartheta_l \right) + \frac{\vartheta_l \left (\vartheta^2 - \vartheta_u^2 \right)}{2\left(\vartheta_u - \pi \right )} + \frac{\vartheta_l \pi \left ( \vartheta_u - \vartheta \right)}{\vartheta_u - \pi}$ and the last partial derivative is
		
		\[
			\left .\frac{\partial T}{\partial q_w}\right |_{t=t_k}  = \scalemath{0.8}{
										\begin{bmatrix}
											-\frac{c_{\vartheta}q_p(t_k)^3 \vartheta_l \left (2\pi-4\arccos(q_w(t_k)) \right) q_x(t_k)}{\left ( \vartheta_u - \pi \right )\sqrt{1-q_w(t_k)^2}} + \frac{c_{\vartheta} q_w(t_k) q_z(t_k) q_p(t_k)^3 \vartheta_l \left (2\arccos(q_w(t_k))-\pi \right ) q_y(t_k)}{\left (\vartheta_u - \pi \right )\sqrt{1-q_w(t_k)^2}^3} - \frac{2 c_\vartheta q_z(t_k) q_p(t_k)^3 \vartheta_l q_y(t_k)}{\left(\vartheta_u - \pi \right )\sqrt{1-q_w(t_k)^2}^2}\\
											-\frac{c_{\vartheta}q_p(t_k)^3 \vartheta_l \left (2\pi-4\arccos(q_w(t_k)) \right) q_y(t_k)}{\left ( \vartheta_u - \pi \right )\sqrt{1-q_w(t_k)^2}} - \frac{c_{\vartheta} q_w(t_k) q_z(t_k) q_p(t_k)^3 \vartheta_l \left (2\arccos(q_w(t_k))-\pi \right ) q_x(t_k)}{\left (\vartheta_u - \pi \right )\sqrt{1-q_w(t_k)^2}^3} + \frac{2 c_\vartheta q_z(t_k) q_p(t_k)^3 \vartheta_l q_x(t_k)}{\left(\vartheta_u - \pi \right )\sqrt{1-q_w(t_k)^2}^2}\\
											\frac{c_{\vartheta} q_z(t_k) q_p(t_k)^4 q_w(t_k) \vartheta_l \left (2\arccos(q_w(t_k))-\pi \right )}{\left(\vartheta_u-\pi \right)\sqrt{1-q_w(t_k)^2}^3} - \frac{2c_{\vartheta} q_z(t_k) q_p(t_k)^4 \vartheta_l}{\left ( \vartheta_u - \pi \right )\sqrt{1-q_w(t_k)^2}^2}
										\end{bmatrix} 
										}
		\]
	\bibliography{thesis_bib}
\end{document}
