\documentclass{article}
	
\usepackage{thesis_style}

\bibliographystyle{ieeetr}
\graphicspath{{./img/}}

\begin{document}
	\title{Notas acerca do código do Arducopter}
	\author{Diogo Almeida}
	\maketitle	
	Onde mantenho notas acerca do código do quadricóptero e onde encontrar o quê.
	
	\section{Variáveis essenciais ao controlador}
		O controlador tem como entrada os três angular rates mais os respectivos ângulos. Como saída deve ser capaz de produzir quatro inputs para os motores. Assim, é necessário:
		\begin{itemize}
			\item IMU angular rates;
			\item Angle estimates;
			\item Controller output function.
		\end{itemize}
	
	As dificuldades que se podem encontrar prendem-se com questões de unidades e sampling rate, assim como de comunicação com controladores de alto nível.
	
	\section{Código disponível}
	Os vários ficheiros de código disponíveis e funções de interesse.
		\subsection{Arducopter.pde}
			Inclui definição do loop principal do sistema e coisas relacionas.
			\subsubsection{Setup()}
				Inicialização do sistema. Não parece ter nada de interessante (talvez notify.init() e init\_ardupilot()).
			\subsubsection{loop()}
				Main loop. Corre o fast\_loop() seguido do scheduler de eventos.
			\subsubsection{fast\_loop()}
				Provavelmente o loop que me interessa. Tem os rate controllers, que creio que terei que substituir pelo código do meu controlador. Preocupante será a estimativa dos ângulos.
			\subsubsection{fifty\_herts\_loop()}
				Onde está o controlo da altitude.
		\subsection{Attitude.pde}
			Tem os vários controladores e funções auxiliares da atitude lá definidos.
		\subsection{AP_Scheduler.h}
			
	
	
\end{document}
